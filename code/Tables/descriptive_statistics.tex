\begin{table}[H]
    \centering
    \caption{Descriptive statistics of variables in model}
    \resizebox{\columnwidth}{!}{
        \begin{tabular}{lccccc}
            \toprule
            {}   & \thead{land price                              \\(1,000 yuan/hectare)}
                 & \thead{output                                  \\ (1,000 yuan/year)}
                 & \thead{number of workers                       \\(person)}
                 & \thead{area of land                            \\ (hectare)}
                 & \thead{city wage                               \\ (1,000 yuan/year)} \\
            \midrule
            mean & 1685.10                  & 170733.30
                 & 228.16                   & 4.59       & 36.96  \\
            std  & 934.95                   & 409499.05
                 & 143.57                   & 5.65       & 18.69  \\
            min  & 379.50                   & 5662.00
                 & 10.00                    & 0.03       & 17.21  \\
            25\% & 978.62                   & 38354.00
                 & 125.00                   & 1.60       & 30.49  \\
            50\% & 1440.01                  & 72909.00
                 & 223.00                   & 2.83       & 34.73  \\
            75\% & 2015.52                  & 164553.00
                 & 324.50                   & 5.40       & 39.83  \\
            max  & 6001.56                  & 7282607.00
                 & 2326.00                  & 65.48      & 320.63 \\
            \midrule
            N    & 1019                     & 1019
                 & 1019                     & 1019       & 288    \\
            \bottomrule
        \end{tabular}
    }
    \label{table: descriptive statistics}
    \fnote{Notes: There are 1019 observations (new firms established in 2012) in our data set.
        The last column is the city average wage for 288 cities,
        which is very close to the total number of
        prefecture-level cities and direct-administered municipalities (297 cities) in China. And
        25\%, 50\%, 74\% in the table mean 25\% percentile, median, 75\% percentile respectively.}
\end{table}