%!TEX root = paper.tex
As discussed in \Cref{sec:intro}, city governments in China use low land prices to
attract industrial firms to produce in their jurisdictions. In this section,
I briefly review the evolution of this kind of fiscal competition and
its impact on China's land market from a historical perspective,
which will provide a basis for the model introduced in \Cref{sec:model}.

\subsection{The evolution of fiscal competition between local governments}
The reason why Chinese local governments compete fiercely against each other
to attract industrial firms is deeply rooted in China's fiscal system,
especially the fiscal revenue sharing scheme between the central government and local governments.
China began its economic reform as a very poor country in 1978,
and the central government used ``fiscal contracting'' system in the 1980s
to incentivize local governments to develop the local economy.
Under this system, local governments got an increasing marginal share of the fiscal revenue
they collected. Moreover, they often colluded with local state-owned enterprises (SOEs),
which generated most of the tax revenue in the 1980s,
to hide the fiscal revenue from the central government \citep{su2017china}.

The ``fiscal contracting'' system led to a continuous decline of the central government's share
of total fiscal revenue and threatened the stability of China's macro economy though it
enriches the local governments \citep{LiuXiong+2020+183+207}. The Tax-sharing Reform in
1994 changed this situation drastically by reconstructing the tax system.
Under this new system, fiscal revenue sharing based on fiscal contracting is abolished,
and the central government takes the larger part of tax revenue collected by local tax bureaus,
which are controlled by the central government directly since then.
The tax revenue left to local governments comprises
25\% of value-added tax (VAT), business tax, and income tax.\footnote{The central government
    also takes 60\% of income tax since 2003.}
Meanwhile, local governments had to accept more obligations for public spending.
To make the matter worse, low-efficient local SOEs were hard to survive under the competition
with private and foreign firms, which are thriving in China since the mid-1990s.
As a result of these dramatic changes in China's fiscal system, local governments
had to find new ways to generate more fiscal revenues to make up for their fiscal shortfall.
The natural choice for local governments is to attract private (including foreign) firms
to land in their jurisdictions.
In addition to the benefits of tax revenue generated by
landing private firms, huge fiscal spillover effects
include the promotion of local businesses since more workers come to the city, and
the land sale revenue promoted by the increasing demand in the residential and commercial housing
market, which will be discussed in the next subsection.
The strong incentives for attracting
firms lead to fiscal competition between local governments.
Selling industrial land at low prices is
the most common method of attracting firms between competing cities.

\subsection{The land market in China}
All urban land in China legally belongs to the state, however, local governments,
especially city and county governments have de facto ownership over land in their
jurisdictions.\footnote{The Land Management Law passed in 1998 authorizes local governments
    to sell usufruct rights over the land. See \cite{LiuXiong+2020+183+207}.}
Thus, local governments are the de facto monopolist of land supply in their jurisdictions
though the upper bound of land supply is dependent upon the land quota set by
the central government.

For residential and commercial land, as the monopolist,
the local governments tend to rise the price by restricting the supply of land
of these two types. According to the national land price index constructed by
\cite{LiuXiong+2020+183+207},
the price of commercial land in China rose from 1 in 2004 to 6.11 in 2015,
and which of residential land rose from 1 to 4.75 during the same period.\footnote{
    The land price index in \cite{LiuXiong+2020+183+207}
    is based on the data
    and calculation of \cite{chen2017real}.}

But for industrial land, the story is different. According to \cite{LiuXiong+2020+183+207},
the price index of industrial land in China was just 1.5 in 2015 (starting from 1 in 2004).
Nevertheless, suppressing industrial land prices is also a rational strategic choice of
local governments. By making special deals with firms, the local governments use low
land prices to attract firms to their jurisdiction.\footnote{The industrial lands
    are sold through case-by-case negotiations and open auctions in China.
    In the first case, the special deal is easily achieved. In the second case,
    which is promoted by the central government to prevent corruption and fiscal revenue waste by
    selling lands at low prices, the special deals can be also realized by sending signals in
    the first stage of the auction to deter the entry of other bidders \citep{cai2013china}.}
And because a firm is
always attracted by several cities, the firm will often get an upper hand in bargaining
with city governments.
Thus, the land price will become lower as the competition becomes fiercer.

Workers always live in the city where they work,
so they don't have too much freedom to choose where to buy houses.\footnote{Actually,
    only a part of workers like advanced engineers and managers buy houses in cities,
    most manufacturing workers coming from rural areas don't buy houses in cities due to
    low wage level and institutional discrimination (the hukou system). And that's why
    the high residential housing price doesn't raise the wage cost of industrial firms and
    influence their location choices.}
And since the service industry (local businesses) thrives in cities
that have many industrial firms and a large consumption market
(which is brought by the large working population), commercial enterprises don't have
too much freedom to choose their locations, either.
Thus, local governments can extract
consumer surplus in both the residential and commercial land markets, which may
not only compensate for their loss in industrial land sales but also bring huge additional
profits since the demand for residential and commercial land
increases sharply if more industrial firms produce in their jurisdictions.
Local governments also get huge fiscal revenue from the promotion of local businesses if more workers live in their jurisdictions.

Having introduced the background of fiscal competition in China,
I build a model of the fiscal competition in the next section.

