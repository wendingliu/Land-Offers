%!TEX root = paper.tex
I present the estimation results under the other two settings of the choice sets $\mathbf{C}_i$ here.

First, instead of \eqref{size of choice set}, I set $|\mathbf{C}_i| = 3$ if
$Y_i \leq Median(Y)$, $|\mathbf{C}_i| = 4$ if $Y_i > Median(Y)$. The estimation results
are shown in \Cref{table: estimates (min_size=3 max_size=4 margin=2000)} below.

\begin{table}[H]
    \centering
    \caption{Estimates of Parameters ($3 \leq |\mathbf{C}_i| \leq 4$)}
    \label{table: estimates (min_size=3 max_size=4 margin=2000)}
    \begin{tabular}{cccc}
        \toprule
        Calibrated $\alpha$ & Parameters & Estimates & Standard Error \\
        \midrule
        0.33                & $\beta$    & 0.537     & 0.046          \\
                            & $\sigma$   & 180.438   & 23.666         \\
        \midrule
        0.5                 & $\beta$    & 0.581     & 0.047          \\
                            & $\sigma$   & 190.297   & 25.614         \\
        \midrule
        0.67                & $\beta$    & 0.580     & 0.048          \\
                            & $\sigma$   & 196.159   & 25.878         \\
        \bottomrule
    \end{tabular}
\end{table}


As \Cref{table: estimates (min_size=3 max_size=4 margin=2000)} shows, the estimates
$\hat{\beta} \approx 0.54, ~0.58, ~0.58$
for $\alpha = 0.33, ~0.5, ~0.67$ respectively.
Thus, $\hat{\beta}$ under this setting are higher than my main results due to the decrease
of the size of choice sets, i.e., the fiscal competition is less fierce than the settings
in the main text, and $\hat{\beta} = \estbeta$ is a conservative estimate.


Second, I set $|\mathbf{C}_i| = 4$ if
$Y_i \leq Median(Y)$, $|\mathbf{C}_i| = 5$ if $Y_i > Median(Y)$. The estimation results
are shown in \Cref{table: estimates (min_size=4 max_size=5 margin=2000)} below.

\begin{table}[H]
    \centering
    \caption{Estimates of Parameters ($4 \leq |\mathbf{C}_i| \leq 5$)}
    \label{table: estimates (min_size=4 max_size=5 margin=2000)}
    \begin{tabular}{cccc}
        \toprule
        Calibrated $\alpha$ & Parameters & Estimates & Standard Error \\
        \midrule
        0.33                & $\beta$    & 0.391     & 0.027          \\
                            & $\sigma$   & 177.253   & 22.551         \\
        \midrule
        0.5                 & $\beta$    & 0.374     & 0.026          \\
                            & $\sigma$   & 196.125   & 24.497         \\
        \midrule
        0.67                & $\beta$    & 0.387     & 0.028          \\
                            & $\sigma$   & 203.421   & 26.099         \\
        \bottomrule
    \end{tabular}
\end{table}


As \Cref{table: estimates (min_size=4 max_size=5 margin=2000)} shows,
the estimates $\hat{\beta} \approx 0.4$
in both cases of $\alpha = 0.33, ~0.5, ~0.67$, which is slightly lower than the main results
in \Cref{table: estimates (min_size=3 max_size=5 margin=2000)}. However, the setting that
half of the firms have 4 candidate cities and the other half have 5 candidate cities
makes the competition for firms very fierce even for a smaller $\beta$, thus, the estimates in
\Cref{table: estimates (min_size=4 max_size=5 margin=2000)} indeed
verifies the robustness of my main results
in \Cref{table: estimates (min_size=3 max_size=5 margin=2000)} since $\hat{\beta}$
is not far away from $\estbeta$ even if the fiscal competition is extremely fierce.

