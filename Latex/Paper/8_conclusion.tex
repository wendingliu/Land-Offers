%!TEX root = paper.tex
In this paper, I have developed and estimated a static Bertrand game model between city governments in China
as they use land sales discounts to attract industrial firms. The estimates show
local governments can get huge benefits from attracting industrial
firms. And the ratio of fiscal revenue to firms' output is much higher
than the official tax rate.

I show that if the total output level in China doesn't change a lot
by fiscal competition, i.e., firms still want to produce in China if the land price
is decided by market power rather than special deals between local governments and firms, the major tool for fiscal competition:
selling industrial land at low prices cannot improve the allocation efficiency of output but waste
potential land-selling revenue. In this sense, my study not only
sheds light on the mechanism of fiscal competition with Chinese characteristics,
but also supports the tax competition model \citep{wilson1999theories}
in the public economics literature.
The counterfactual analysis also shows
the decline of the economic autonomy of local governments poses potential challenges to the mode of fiscal competition.
But rising wages might not have significant influence on the fiscal competition pattern.

However, my study doesn't imply the fiscal competition between city governments in
China is worthless. The real potential benefits of fiscal competition can come
from the case that more firms (or higher output levels) are created due to the
lower production cost induced by the efforts of local governments to
attract firms. But this issue is beyond the scope of my analysis since the firm's output level
is exogenous in the model, and I don't consider the entry and exit problems of firms.
Analyzing these issues can be a possible extension of the paper.

Incorporating a labor market into my model might be another interesting extension
since local governments can use policy instrument
(e.g. household registration system reform, enhancing social welfare, etc.)
to attract workers to their jurisdictions and decrease the production cost of firms indirectly.
And I leave these two possible extensions for further research.

