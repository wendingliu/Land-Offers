%!TEX root = paper.tex
In this paper, I analyze the fiscal competition between city governments in China as they use
industrial land sales discounts to attract industrial firms to
land in their geographical jurisdictions.
Understanding this process is important since
the unique ``regionally decentralized authoritarian regime'' \citep{xu2011fundamental}
plays an important role in China's economic growth.
This regime is unusual in combining a high degree of political centralization
and economic decentralization, and it gives local officials great autonomy
in economic and fiscal issues (\citealp{kroeber2020china,xiong2018mandarin}).
Thus, local officials have strong
incentives to develop the local economy as well as maximize their fiscal revenues.
One of the most salient characteristics of the system is the
competition among city governments, which use special deals to
attract businesses \citep{bai2020special}. A major type of special deals is
selling industrial lands at low prices to firms \citep{su2017china}.
Thus, my study helps to understand the economic
system of China, and it also sheds light on broader public economics issues, especially the mechanism
and impacts of fiscal competition caused by the decentralization of economic and fiscal power.

Local governments can generate fiscal revenue by landing industrial firms.
The fiscal revenue includes tax revenue, promotion of local businesses
and housing market, political benefits, etc. However, most of them except the tax revenue cannot
be measured directly from the data. Thus, the total fiscal revenue generated by
landing industrial firms is difficult to be measured directly.
I address this issue by structurally estimating a Bertrand pricing game model
that characterizes the fiscal competition between Chinese city governments.
In this model, city governments maximize their fiscal revenue by
providing special industrial land price offers to attract firms, and they need to consider
competitor city governments' strategies when they make land price offers. The basic
trade-off in the model is between the higher probability of getting the firm
and higher land-selling revenue.
I solve the continuous Bertrand pricing game using the Gauss-Seidel algorithm,
and structurally estimate the key parameters of this model by the method of simulated moments (MSM),
which is numerically implemented by a polyalgorithm combining the grid search and Quasi-Newton method.

The estimation results show city government can generate a huge amount of fiscal revenue from
landing industrial firms.
The total fiscal revenue is around \estbetapercent ~of the firm's
per year output level, which is far beyond the official tax revenue.
This finding shows the fiscal spillover effects
of landing industrial firms for city governments are enormous,
and that's why local governments in China compete against each other so fiercely
to attract industrial firms.
I use counterfactual experiments to study the impacts of this fiscal competition on resource allocation,
and the results suggest the impact is small assuming the total output level is fixed.
I also simulate the impacts of fiscal power re-centralization and rising wages on this kind of
fiscal competition.
The simulation results show that both fiscal power re-centralization and the rising urban wage
will raise industrial land prices moderately.

A large strand of literature studies the decentralization of economic and fiscal power in China,
particularly,
\cite{cheung2014economic}, \cite{su2017china}, \cite{bai2020special}
and \cite{LiuXiong+2020+183+207}
all notice that local governments in China use land sales discounts to attract firms.
\cite{chen2017real} constructs a land price index in China and finds that price of
industrial land in China is significantly lower than commercial and residential land.
\cite{bai2020special} also confirms the fact by regression analysis.
But these papers don't build any formal economic model to explain this issue.
For general theoretical discussion on the impacts of fiscal competition,
literature built on \cite{tiebout1956pure} emphasizes the welfare improvement effect caused by
competition for mobile capital, which creates efficient equilibrium.
However, literature on the tax competition models (\citealp{keen1997fiscal,wilson1999theories})
emphasizes the downward pressure on fiscal revenue induced by fiscal competition and
the possibilities of ``race to the bottom''.
See \cite{wilson1999theories} for a literature survey for this research area.

For the structural estimation, my work is closest to \cite{mast2020race},
which estimates a Bayesian game between towns in the U.S.
as they use tax breaks to bid for firms.
Two key differences are that I explicitly model the cost minimization problem of the firms, and I solve
the game on a continuous space rather than on grids. The numerical methods for solving the game and
estimation are both fast.\footnote{
    It costs around 15 minutes to get the main estimation results in \Cref{sec:results}
    on the author's PC with Core i7-1165G7 (2.80GHz) CPU and 16 GB RAM.}

The outline of the paper is as follows: In \Cref{sec:institution} I examine the evolution
of fiscal competition between local governments in China and its relation with the land market.
\Cref{sec:model} presents my model and the numerical method to solve the model.
\Cref{sec:data} and \Cref{sec:estimation} describe the data and my estimation method.
\Cref{sec:results} shows the estimates and the fit of my model.
\Cref{sec:counterfactuals} shows the counterfactual analysis based on the estimates.
And \Cref{sec:conclusion} concludes.


