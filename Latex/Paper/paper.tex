\documentclass[11pt, english]{article}
\usepackage{enumerate}
\usepackage{graphicx}
\usepackage{caption}
\usepackage{float}
\usepackage{amsmath, mleftright}
\DeclareMathOperator*{\argmax}{argmax}
\DeclareMathOperator*{\argmin}{argmin}
\DeclareMathOperator{\Var}{Var}
\DeclareMathOperator{\AVar}{AVar}

\usepackage{xparse}
\usepackage{mathtools, nccmath}
\usepackage{algorithm2e}
\usepackage[T1]{fontenc}
\usepackage[latin9]{inputenc}
\usepackage[a4paper]{geometry}
\geometry{verbose,tmargin=1in,bmargin=1in,lmargin=1in,rmargin=1in}
\setlength{\parskip}{\smallskipamount}
\setlength{\parindent}{2em}
\usepackage{amsmath,amsthm,amsfonts,amssymb}
\usepackage{setspace}
\usepackage{hyperref}
\usepackage[dvipsnames]{xcolor}
\definecolor{airforceblue}{rgb}{0.36, 0.54, 0.66}
\hypersetup{linkcolor=airforceblue,citecolor=airforceblue,urlcolor=black,
filecolor=red,colorlinks=true}

% \cref (for lower case) and \Cref (for upper case)
\usepackage[nameinlink]{cleveref}

% use \Cref to cite equation
\Crefformat{equation}{#2Eq.~(#1)#3}

% change \eqref to (1), where the parentheses are colorful
\makeatletter
\renewcommand*{\eqref}[1]{%
  \hyperref[{#1}]{\textup{\tagform@{\ref*{#1}}}}%
}
\makeatother


% tables
\usepackage{booktabs}
\usepackage[normalem]{ulem}
\useunder{\uline}{\ul}{}

\usepackage{makecell}

\usepackage{etex}

%Bibtex references
\usepackage[round,nonamebreak]{natbib}
% \bibliographystyle{plainnat}
\bibliographystyle{econometrica}
\usepackage{babel}

\usepackage{datetime} % month-year format
\newdateformat{monthyeardate}{%
  \monthname[\THEMONTH], \THEYEAR}

\theoremstyle{plain}
\newtheorem{prp}{Proposition}
\newtheorem{thm}{Theorem}
\newtheorem{cor}{Corollary}[thm]
\newtheorem{lem}{Lemma}
\theoremstyle{definition}
\newtheorem{defn}{Definition}
\newtheorem{assump}{Assumption}
\renewcommand\theassump{A\arabic{assump}}

\newtheorem{innerassumprim}{Assumption}
\newenvironment{assumprim}[1]
	{\renewcommand\theinnerassumprim{#1}\innerassumprim}
	{\endinnerassumprim}


\newcommand{\RomanNumeralCaps}[1]{\MakeUppercase{\romannumeral #1}}
\newcommand{\RomanNumeral}[1]{{\romannumeral #1}}


\newcommand{\bi}{\begin{itemize}}
\newcommand{\ei}{\end{itemize}}
\newcommand{\bn}{\begin{enumerate}}
\newcommand{\en}{\end{enumerate}}

% footnote of the graphs
\newcommand\fnote[1]{\captionsetup{justification=raggedright, singlelinecheck=false,
font=footnotesize}\caption*{#1}}

% \renewcommand{\epsilon}{\varepsilon}	 % OVerwrites \epsion with \varepsilon

\newenvironment{tight_enumerate}{
\begin{enumerate}[(i)]
  \setlength{\itemsep}{0pt}
  \setlength{\parskip}{0pt}
}{\end{enumerate}}
\newcommand{\bnt}{\begin{tight_enumerate}}
\newcommand{\ent}{\end{tight_enumerate}}

%partial derivative
\NewDocumentCommand{\evalat}{sO{\big}mm}{%
  \IfBooleanTF{#1}
   {\mleft. #3 \mright|_{#4}}
   {#3#2|_{#4}}%
}


\tolerance=2000
\vbadness=10000
\hbadness=10000
\def\argmin{\mathop{\it argmin}}
\def\argmax{\mathop{\it argmax}}
% to discourage line breaks in math
\binoppenalty=10000
\relpenalty=10000
\interfootnotelinepenalty=10000 % Completely prevent breaking of footnotes

% graphs and tables paths
\newcommand{\graphs}{../../Graphs}
\newcommand{\tables}{../../Tables}

\newcommand\estbeta{0.45}
\newcommand\estbetapercent{45\%}
\newcommand\estsigma{178}


%% BEGIN DOCUMENT
\begin{document}

\title{Land Offers and Fiscal Competition Between City Governments in China\thanks{
    I am
    indebted to my supervisor Fedor Iskhakov for
    his continuous guidance on this paper.
    I acknowledge comments from Xin Meng, Bob Gregory, Ruitian Lang, and John Stachurski
    at the applied microeconomics seminar at ANU.}
}

\date{\monthyeardate\today}

\author{Wending Liu\thanks{ Australian National University
    \href{mailto:Wending.Liu@anu.edu.au}{\texttt{Wending.Liu@anu.edu.au}}}
}
\maketitle

\noindent \begin{center}
  \par\end{center}

\begin{abstract}
  I analyze the fiscal competition between city governments in China
  by structurally estimating a Bertrand pricing game model.
  The model characterizes the land pricing strategy of city governments as they use land
  sales discounts to attract industrial firms.
  The estimation results imply that city governments can generate a huge amount of fiscal revenue
  from landing industrial firms,
  which is around \estbetapercent ~of the firm's yearly output.
  By counterfactual experiments, I show that
  the impact of this kind of fiscal competition on resource allocation
  is small. Simulation results also show that fiscal
  centralization and increasing urban wages would result in a modest
  average land price increase.


  \bigskip{}


  \noindent \textbf{Keywords:} Fiscal competition, land market, Bertrand game,
  structural estimation

  \smallskip{}


  \noindent \textbf{JEL codes:} C57, H71

  \bigskip{}

\end{abstract}

\clearpage{}

\begin{spacing}{1.5} % 1.5 line space

  %The structure of the paper:
  \section{Introduction}
  \label{sec:intro}
  \input 1_intro.tex

  \section{Fiscal Competition and Land Market in China}
  \label{sec:institution}
  \input 2_institution.tex

  \section{The Model}
  \label{sec:model}
  \input 3_model.tex

  \section{Data}
  \label{sec:data}
  \input 4_data.tex

  \section{Estimation Method}
  \label{sec:estimation}
  \input 5_estimation.tex

  \section{Estimation Results}
  \label{sec:results}
  \input 6_results.tex

  \section{Counterfactual Analysis}
  \label{sec:counterfactuals}
  \input 7_counterfactuals.tex

  \section{Conclusion}
  \label{sec:conclusion}
  \input 8_conclusion.tex

\end{spacing}


\begin{spacing}{1.5}

  \section*{Appendix A: Proofs of the Theorems}
  % add this section to the table of content
  \label{sec:proofs}
  \addcontentsline{toc}{section}{\protect\numberline{}Appendix A: Proofs of the Theorems}
  \input 9_appendix_a.tex

  \section*{Appendix B.1: Estimation Under Different Settings of Choice Sets}
  \addcontentsline{toc}{section}{\protect\numberline{}Appendix B.1: Estimation Under
    Different Settings of Choice Sets}
  \label{sec:robustness checks 1}
  \input 10_appendix_b1.tex

  \section*{Appendix B.2: Estimation Under Different Participation Constraints}
  \addcontentsline{toc}{section}{\protect\numberline{}Appendix B.2:
    Estimation Under Different Participation Constraints}
  \label{sec:robustness checks 2}
  \input 11_appendix_b2.tex






  %References

  \addcontentsline{toc}{section}{References}
  \bibliography{reference}


\end{spacing}

\end{document}

